\documentclass{amsart}

% my notation
\newcommand{\col}{\chi}
\newcommand{\colt}{\tilde{\chi}}
\newcommand{\symmdiff}{\Delta}
\newcommand{\colorS}{\mathcal{C}}
\newcommand{\subcolS}{\mathsf{Sub}}
\newcommand{\treecut}{\Psi}
\newcommand{\cut}{\kappa}
\newcommand{\bad}{\beta}

% my operator names
\newcommand{\rk}{\operatorname{rk}}
\newcommand{\subrk}{\operatorname{subrk}}
\newcommand{\minsubrk}{\subrk_{\min}}
\newcommand{\mrca}{\sup}
\newcommand{\opt}[1]{\overset{\circ}{#1}}
\newcommand{\mlst}{\operatorname{mlst}}
\newcommand{\Bell}{\operatorname{Bell}}

% others' notation
\newcommand{\nbadcolors}{n_c^*}
% theorems, etc
\newtheorem{lemma}{Lemma}
\newtheorem{prop}{Proposition}
\newtheorem{theorem}{Theorem}
\newtheorem{problem}{Problem}
\newtheorem{defn}{Definition}


\begin{document}

% \section{Introduction}
% explain induced subtree and rooted subtree

\section{Finding a tree that does not violate taxonomic assignments}
Define minimal deletion to be the minimal number of taxa that have to be deleted in order to have a tree that is concordant with the taxonomic assignments.

\subsection{Agreement with a complete taxonomic tree}
This is the same as MCST, and is hard (french)
We may not want it anyway.

\subsection{Agreement with leaf taxonomic assignments}

Define an ``color set'' as a set of colors.
Define coloring of a given tree to be a mapping from a subset of the nodes of the tree to the colors.
A leaf coloring is a coloring with domain a subset of the leaves.
A multicoloring is a map from the edges of the tree to subsets of the colors.
We say that a color $c$ is cut by an edge $e$ if $e$ lies in the subtree induced by the leaves of color $c$.

Given a leaf coloring on a tree $T$
\[
\col: F \subset L(T) \rightarrow C
\]
define the induced multicoloring
\[
\colt: E(T) \rightarrow 2^C
\]
such that $\colt(e)$ is the (potentially empty) set of colors which are cut by that edge.

Define the badness $\bad(\col)$ of a coloring $\col$ to be $\max_{e \in E(T)} |\colt(e)| $ for all $e$.
We say that a coloring is convex if it has badness equal to zero or one.

\begin{defn}
  A subcoloring of a leaf coloring $\col: F \subset L(T) \rightarrow C$ is a coloring $\nu: G \subset F \rightarrow C$ such that $\nu$ agrees with $\col$ on the domain of $\nu$.
  Let $\subcolS(\col)$ be the set of subcolorings of a given coloring.
\end{defn}
These are partially ordered by inclusion of domains.
The size of a subcoloring is defined to be the size of its domain.

For two colorings $\col_1$ and $\col_2$ with disjoint domains we can make a coloring on the union of their domains in the obvious way, which will be denoted $\col_1 \sqcup \col_2$.

\begin{problem}
\label{prob:subcolor}
  Given a leaf coloring $\col$ on a tree $T$, find a largest convex subcoloring in $\subcolS(\col)$.
\end{problem}

\subsubsection{Hardness}
Given a set of leaf taxonomic assignments, we would like to find the maximum-sized set of taxa which is compatible with the phylogenetic tree.
Call this the maximum agreement subset of taxa.

We can reduce it to a minimal recoloring in the sense of Moran and Snir.
Assume we are given a tree with leaf coloring, and we know a minimal leaf deletion.
Then we can construct a minimal leaf recoloring by just taking the color of the first ancestor for each deleted leaf.

\begin{prop}
  A minimal deletion and ancestral leaf recoloring gives a minimal leaf recoloring.
\end{prop}

\subsubsection{Fixed parameter tractability}
% Using their more general framework, \cite{moranSnirRecoloringApprox05} have shown that the problem admits a FPT solution.
Their solution is $O(n^4 \nbadcolors \Bell(\nbadcolors))$, where $\nbadcolors$ is the number of ``bad'' colors, i.e. the number of colors which violate convexity (if we take the induced coloring on the tree, then cut by it, then $\nbadcolors$ is the number of colors which exist in more than one connected component).
Note that this algorithm is polynomial in $n$ and exponential in the total number of colors which violate convexity.

In this section we describe a fixed parameter algorithm that is exponential in a local quantity rather than the global number of bad colors.
It is not a general improvement over their bound, but rather a better bound for the more specific problem investigated here.


\subsubsection{Recursion}

\begin{defn}
  Let $(S, \col)$ be a colored subtree and let $T$ be a rooted subtree of $S$.
  Define $\cut(T)$ to be the colors cut by the root edge of $T$ as it sits inside $S$.
\end{defn}

\begin{defn}
  Given a rooted subtree $T$ of a rooted colored tree $(S, \col)$, define a subsolution for a rooted subtree $T$ and $X \subset \cut(T)$ to be a maximal element $phi(X)$ of $\subcolS(\col)$ which includes at least one taxon with each color in $X$ and such that the domain of $\phi(X)$ is a subset of $L(T)$.
\end{defn}

\begin{defn}
  Define a solution map for a rooted subtree $T$ of a rooted colored tree $(S, \col)$ to be a map
  \[
  \phi: 2^{\cut(T)} \rightarrow \subcolS(\col)
  \]
  to be a map from subsets of $\cut(T)$ to their correspoinding subsolutions.
\end{defn}


\begin{prop}
Problem~\ref{prob:subcolor} can be solved for a rooted colored tree $(S, \col)$ with $n$ leaves in $O(n 2^{\bad(\col)})$ time.
\end{prop}

\begin{proof}

The proof will proceed by constructing solution maps for all subtrees $T$ of $S$ by recursion.
In fact, we will show that the solution map for a given internal node can be constructed from the maps for the nodes below it in time $2^{\bad(\col)}$;
this will demonstrate the proposition because there are $O(n)$ internal nodes and the leaf solution maps are trivial.

For the recursion, assume that subtree $T$ is composed of two rooted subtrees $T_1$ and $T_2$.
The correctness of the recursion is simple: every optimal solution can be broken down into elements of solution maps for the subtrees.
Let $K = \cut(T)$, $K_1 = \cut(T_1)$ and $K_2 = \cut(T_2)$.
By the recursion, these rooted trees $T_1$ and $T_2$ have solutions $\phi_1$ and $\phi_2$, respectively.

We would like to find a solution $\phi(X)$ for any $X \in 2^K$.
Everything depends on deciding $X_1 \subset K_1$ and $X_2 \subset K_2$ such that $X = (X_1 \cup X_2) \cap K$.
Once these choices are made the value of $\phi(X)$ will be $\phi(X_1) \sqcup \phi(X_2)$.
It is required that $|X_1 \cap X_2| \leq 1$ to ensure convexity.
The optimal choice of $X_1$ and $X_2$ will be the one that maximizes the size of $\phi(X_1) \sqcup \phi(X_2)$.
Every optimal solution can be expressed in this form by definition.

Rather than looking for an optimal solution for a given $X$, we will evaluate every legal $X_1$ and $X_2$ and assign the value of that pair to the corresponding $X$.
If $X_1$ and $X_2$ are disjoint, then set $X$ to be $(X_1 \cup X_2) \cap K$: anything in them can be used.
If $X_1$ and $X_2$ intersect in a single element $c$, then set $X$ to $\{c\} \cap K$.
Note that in this latter case we have $X_1 = X_2 = \{c\}$.

A naive implementation of this recursion will check every subset of $X_1$ and $X_2$; therefore every node is going to have complexity $O(2^{2 \bad(\col)})$.
This is sub-optimal.
Indeed, for cases when $X_1$ and $X_2$ are disjoint, we can divide the possible $X$ into $X_1 \symmdiff X_2$ and $X_1 \cap X_2$.
The elements of $X_1 \symmdiff X_2$ come ``for free'': we can add them or take them away at will.
Thus we need to find the optimal partition of $X_1 \cap X_2 \cap K$ (for the $\cap K$ we don't need to worry about things which are only common to $X_1$ and $X_2$ and don't appear other places); this is $O(2^{\bad(\col)})$ but shouldn't be bad.
One can cover the cases when $X_1$ and $X_2$ are not disjoint by iterating $X$ through the single-element sets of $X_1 \cap X_2$.

\end{proof}

% This corresponds with the original definition of $X$:
% \begin{lemma}
% \[
%   (X_1 \cup X_2) \cap K = (C_1 \cup C_2) \cap K
% \]
% \end{lemma}
% \begin{proof}
%   $\subset$ is clear.
%   For $\supset$, say a $c \in C_1 \cap K$.
%   Thus $c \in L(T) \setminus (C_1 \cup C_2)$, and thus $c \in K_1$.
% \end{proof}

% $C_1 \cap C_2 \subset K_1 \cap K_2$ because anything in both is in each of the cut sets.
% $C_1 \cap C_2 \supset K_1 \cap K_2$ because anything in each of the cut sets must be in each of $C_1$ and $C_2$.
% So they are equal.
% Thus $X_1 \cap X_2 = C_1 \cap C_2$.


\section{Computational matters}

Rerooting at the max $K$ will reduce the amount of computation.

\section{Taxonomic rerooting}

Given a taxonomic identifier $u$, let $\rk(u)$ be the rank of $u$ in the taxonomic hierarchy.
Given a set of taxonomic identifiers $U$, let $\mrca(U)$ denote the most recent common ancestor of the identifiers in $U$.
By an abuse of notation, we also let $\rk(U)$ signify $\rk(\mrca(U))$.

\begin{lemma}
  \[
  \rk(A \cup B) \geq \max(\rk(A),\rk(B))
  \]
\end{lemma}

Given $x$ a node of $T$, say $\phi(x;T)$ are the trees obtained by deleting $x$ from $T$.
Define $\subrk(x;T)$ to be $\max_{S \in L(\treecut(x;T))} \rk(S)$.

Let
\[
\minsubrk(T) = \min_{x \in N(T)} \subrk(x;T)
\]

Given $x \in N(T)$, let ${A_1,\cdots,A_n} = \treecut(x;T)$, and that
\[
  (a_1,\cdots,a_n) = (\rk(A_1), \cdots, \rk(A_n))
\]
assume WLOG that $a_1 \leq a_2 \leq \cdots \leq a_n$.

There are two cases: $a_{n-1} = a_n$ and $a_{n-1} < a_n$.
In the first case, we know that $\minsubrk(T) = a_{n-1} = a_n$ because any direction we travel will leave one of the $a_i$'s intact, which will act as a lower bound for $\minsubrk(T)$.
In the second case, $a_{n-1} \leq \minsubrk(T)$ but $\minsubrk(T)$ but may be less than $a_n$, which we can discover by moving towards it.


\end{document}

% leftovers from cleanup

\begin{defn}
  Let the coloring set $\colorS(T, C)$ where $T$ is a phylogenetic tree and $C$ is a set of colors to be the set of maps $\chi: X \rightarrow C$ where $X \subset L(T)$.
\end{defn}


Kicked out:
\begin{enumerate}
  \item $r_1 \in X_1 \cap X$ and $r_2 \in X_2 \cap X$
(this is too strong: we are now not worrying about the root ``state'' unless it matters)
\end{enumerate}

Don't forget!
$X_1$ and $X_2$ need not sit in $X$.




Let $X^c$ be $X$'s complement in $K$.


\subsection{Notation}
$\symmdiff$ is the symmetric difference.

The sets of interest are
$I_i = K \cap K_i$
$J_1 = I_1 \setminus K_2$ and $J_2 = I_2 \setminus K_1$. (these are ones which can only show in one)
$O = (K_1 \cap K_2) \setminus K$ (these are the ones which can only arc over)

Explicit descripton:
We start with our $X$.
We take every $O_1$ and $O_2$ subsets of $O$ such that $|O_1 \cap O_2| \leq 1$ and $(O_1 \cup O_2) \cap X^c = \emptyset$.
We also need to make sure that we don't have $O_1$ and $X$ intersecting with a different color when $O_1$ and $O_2$ do.
Then $X_i = J_i \cup O_i$.

The solution for $X$ is then the union of the best for each $X_i$.

Should do this functorially-- because we might be able to get away with bit strings later.

Two choices of where the color of $(T_1, T_2)$ MRCA root color comes from:
- $R = K_1 \symmdiff K_2$, in which case $X_1$ and $X_2$ are disjoint, which gives an $|R| 2^{|R|}$.
- $U = X \cap K_1 \cap K_2$, in which case we have $|U| 2^{|U|}$.

So that could be a $2^{|K_1 \cap K_2|}$ operation.
We have to do this for every $(\rho, X)$, meaning that the whole machine will have complexity $K \times 2^{|K| + |K_1 \cap K_2|}$.

Let $\kappa$ be the maximum across edges $e$ of $\|\chi(e)\|$.
Complete algorithm: $O(n \kappa 2^{|\kappa| + |\kappa_1 \cap \kappa_2|})$

We are taking the max over distributions of colors, and for each distribution we are taking the best solution.
Every legal distribution of colors follows our specified rules.

Any rooting will work.
Do we get better performance from an intelligent rooting?
That way, we will have smaller taxon numbers when we have to investigate lots of choices.

Data structures:
Want to have a map from root color options and subsets of K to optimal solutions.
That will be an actual map from elements cross subsets to sets.
Could we have a durable subset thingy?

There is a partial order for these things?
Can we restrict the domain a bit?


We can ta

Start by finding cut colors for each split

Wouldn't it be easier to disallow colors?
This color is already used on the other side of the tree, and adding it would change some chosen root color

Disallowing colors:

\begin{defn}
  Define a solution map for a sub-pair $(T,K)$ of a colored rooted tree $(S, \col)$ to be a map
  \[
  \phi: \opt{K} \times 2^K \rightarrow \subcolS(\col)
  \]
  from a choice of the root color and a choice of excluded colors below the root edge to a rooted solution.
\end{defn}

That is, $\phi(r, C)$ a set of taxa in the tree, including only taxa with the color C
such that the induced coloring is convex and the taxon set is maximal among subsets with these color characteristics.

Now, say that tree is formed of two subtrees which have sub-pairs $(T_1,K_1)$ and $(T_2,K_2)$ and complete rooted solutions for each.
Say we would like to find a solution for $(\rho, X) \in \opt{K} \times 2^K$.
If a color is in
Let $R = X \cap (K_1 \symmdiff K_2)$; for these colors we are Really sure where they should come from.
We take every $X_1,X_2$ pair such that
- $X_1 \subset K_1$
- $X_2 \subset K_2$
-
why would we exclude something that is not cut?

Let $U = X \cap K_1 \cap K_2$; these are the colors which we don't know which tree to take them from.
Note that $K_1 \cap K_2$
For every $X_1 \subset K_1$ and $X_2 \subset K_2$ such that $K \ (X_1 \cup X_2) \subset \{\rho\}$ and $X_1 \cap X_2 = X$, we find the solution such that $X \cup X_1$ is excluded in $T_1$ and $X \cup X_2$ is excluded in $T_2$.
So don't we have to keep track of the exclusion of every possible bad color subset?
No, because if a color isn't cut by a
Some exclusions are trivial, because we just take the
At the end, each side is going to have to have exclusions for every single bad color.
The solution for that $X$ is the merge of the best solutions for subsets $X_1 \subset K_1$ and $X_2 \subset K_2$ such that $K \ (X_1 \cup X_2) \subset \{\rho\}$ and $X_1 \cup X_2 = X$.
So that could be a $2^{|K_1 \cap K_2|}$ operation.
We have to do this for every $(\rho, X)$, meaning that the whole step will have complexity $K \times 2^{|K| + |K_1 \cap K_2|}$.


% A weaker but easier to state thing would be to have maps to subsets:
% Define the Constrained Maximum Agreement Taxon Subset (CMATS) problem for a given $(\chi, e, c)$ where $\chi$ is a leaf coloring and $e$ is an edge and $c \in \chit(e)$ be a taxon set $X$ of maximal size such that $\chi$ restricted to $X$ is convex and $\chi|_X (e) = {c}$.
% The diadvantage is that then we fix the original coloring as a global variable, and the range of a subcoloring is then defined with respect to that.

% \bibliography{algotax}
% \bibliographystyle{plain}

